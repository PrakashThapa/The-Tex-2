All the XQuery expressions are translated into NoSQL database queries based on the output result. In case of availability of secondary indexes, they have been created and utilized for efficient result. There are some problems for following queries:
\todo{why not Q4?}
\begin{itemize}
\item Q4 and Q7 cannot be applied to NoSQL databases. These queries will be skipped during the measurement and analysis. In Q7, XPath  wild cards cannot be used in NoSQL databases, which makes it impossible to perform the query. 

\item In Q14, full text search is measured in XQuery. MongoDB, Couchbase and RethinkDB do not support full text search, therefore  sub-string search is performed in these databases. 
\end{itemize}

\subsection{XMark Queries in MongoDB}

In MongoDB, all the possible query methods have been implemented to get the same result as of XQuery.  For an example, the \textit{find()} function can be used for Q1 to get result. As we can see in  Code~\ref{mongo-xmark-q1}, the collection is filtered with default index \textit{\_id}  and the second parameter specifies the \textit{name} field to be retrieved.  For advanced queries, the aggregation pipeline and mapreduce are used. 

As mentioned in ~\ref{nosql-mongodb}, MongoDB has no support for  join queries. All the queries that need more than one collection  Q8, Q9, Q10, Q11 and Q12  are processed through \textit{mongo shell}. Secondary indexes are created only when queries can be optimized. Without indexes, MongoDB has to scan every document in a collection to match the criteria of the query. MongoDB secondary index can be created by using \textit{createIndex} function as in Code~\ref{mongodb-create-index}. Some of the important queries are explained here:

\begin{itemize}
\item Q1 implements the default index \textit{\_id}. Firstly, the collection is filtered then \textit{name} field is specified for selection. 

\item For Q2 and Q3, two indexes in fields \textit{initial} and \textit{"bidder.increase"} of collection \textit{open\_auctions} are created. With various steps of aggregation pipeline the result is returned.

\item A comparison operator is implemented in Q5 that can be easily optimized with index. A secondary index is created in field \textit{price} of \textit{closed\_auctions} collection. 
\item According to our data model for MongoDB, Q6 counts the number of documents in the collection \textit{regions}.
\item In Q8, joins between two collections \textit{people} and \textit{closed\_auction} are performed. The field \textit{buyer.person} is  indexed, therefore the \textit{person} search is efficient. 
The Q9 implements index of Q8 and a new index on \textit{regions} field of \textit{regions} collection is added which is also utilized in Q13.
\item For Q10, another index on \textit{profile.interest} field of \textit{people} collection is created. A helper function is used to join and format the data.
\item For Q14, a \textit{text index} is created in a collection \textit{regions} that allow to search by any sub-strings in whole collection. 
\item The result of Q15 and Q16 do not match the exact result of XQuery, because  XQuery uses  XPath whereas MongoDB do not.

\item The quey Q18 implements user-defined function. In MongoDB, it has to process two steps: A function is created and stored in database system document. Then, it is called through mapreduce. Table~\ref{tbl:mongodb-q18} illustrates Q18 using mapreduce.

\item For Q19, a secondary index on field \textit{location} of collection \textit{regions} is created.
It is used for sorting the data by the \textit{location}. 
\item
The final query Q20 based on aggregations. This query in mapreduce  is much simpler than aggregation pipeline.
\end{itemize}
Full list of MongoDB queries and index can be found in ~\ref{mongodb-query-list}

\begin{figure}
\centering
\begin{lstlisting}[language=JSON, caption=XMark Query Q1 in MongoDB, label=mongo-xmark-q1]
		db.people.find({_id:"person0"},{_id:0,"name":1});
\end{lstlisting}

\centering
\begin{lstlisting}[language=JSON, caption=MongoDB secondary Index, label=mongodb-create-index]
             db.closed_auctions.createIndex({regions:1})
\end{lstlisting}
\end{figure}



\begin{longtable}{c|c}
    \hline
	\caption{ User-defiend function and implementation in MongoDB(Q18)}
	\label{tbl:mongodb-q18}\\
    {function } & {map-reduce}\\
	\hline
\begin{minipage}{.4\textwidth}
\begin{lstlisting}[language=JSON,basicstyle =\scriptsize]
    db.system.js.save({ 
        "_id": "reserve", 
        "value": 
            function(a){ 
                return 2.20371*a; 
            } 
    })
\end{lstlisting}
\end{minipage} &
\begin{minipage}{.4\textwidth}
\begin{lstlisting}[language=JSON,basicstyle =\scriptsize]
db.open_auctions.mapReduce(
    function() {
       if(this.reserve){
        emit(this._id, reserve(this.reserve));
       }    
    },
    function(key,values) {
        return Array.sum(values);
    },
    { "out": { "inline": 1 } }
 );
\end{lstlisting}
\end{minipage}
\end{longtable}


\subsection{XMark Queries in Couchbase}
 In Couchbase, everything is queried with the views. The output of the views are queried through programming interface. For each query, the \textit{doctype} should be checked.  All XMark queries have been queried and evaluated through Node.js in Couchbase. Some of the important queries are illustrated here:
 
 \begin{itemize}
 \item Q1 is a simple query in which the map function emits \textit{id} and \textit{name} of document if the condition is matched.
 \item Q2 and Q3 are deeper array and object processing. Map function use JavaScript code to find the required objects or arrays elements to emit them. 
 \item Both Q5 and Q6 use the built-in \textit{\_count} in reduce section.
 \item The join operation in Couchbase is bit complicated as compare to other NoSQL database. Each \textit{doctype} should have a view and it can be implemented multiple times. For example, in case of Q11 and Q12, view on 
 \textit{open\_auctions} can be used for both queries.  The join operation is completed using programming interface. The Q8 contains two views, one selects the \textit{id} and \textit{name} field from \textit{people}. other view aggregates \textit{doctype} \textit{closed\_auctions} using reduce function with query option \textit{group\_level=1}. Finally, both views are combined using Node.js. 
 \item In Q14, the \textit{description} field in \textit{regions} are converted into string  that makes substring search possible.
 \newpage
 \item In Q18, for a user-defined function implementation, a JavaScript function is created at map section and called it during emit. 
 \item In Q19, a query parameter \textit{descending} set to \textit{false} for result in ascending order. 
 \end{itemize}
 

 Table~\ref{tbl:couchbase-q20} illustrates a sample example of XMark  Q20 with  mapreduce functions and the query parameters. The full list of Couchbase Mapreduce can be found in ~\ref{couchbase-query-list}.

\begin{longtable}{c|c|c}
	\caption{ XMark query Q20 in Couchbase Server}
	\label{tbl:couchbase-q20}\\
    {map} & {reduce} & {query}\\
	\hline
\begin{minipage}{.5\textwidth}
\begin{lstlisting}[language=JSON,basicstyle =\scriptsize]
function (doc, meta) {
    if(doc.doctype=="people"){
      var income = (doc.profile && doc.profile.income) ? 
                doc.profile.income : 0;
      if(income >= 100000 ){
    	 emit("preferred",1);
      }else if(income < 100000 && 
               income >= 30000) {
        emit("standard",1);
      }else if(income < 30000 &&
           income > 0 ){
       
        emit("challenge",1);
      } else {
       emit("na",1);
      }
    }
  }
\end{lstlisting}
\end{minipage} &
\begin{minipage}{.15\textwidth}
\begin{lstlisting}[language=JSON,basicstyle =\scriptsize]
     _sum
\end{lstlisting}
\end{minipage} &
\begin{minipage}{.2\textwidth}
\begin{lstlisting}[language=JSON,basicstyle =\scriptsize]
     group_level=1
\end{lstlisting}
\end{minipage}
\end{longtable}

\subsection{XMark Queries in RethinkDB}

In RethinkDB, the performance of a read query can be improved through the secondary indexes. For XMark queries,  wherever possible, these indexes are utilized. The index has to be defined in the query and can be used only in one of the four functions \textit{getAll()}, \textit{between()}, \textit{eqJoin()} and \textit{orderBy()}. Table~\ref{tbl:rethinkdb-index-query} illustrates an example of creating a secondary index and its usage in a query.
\begin{longtable}{c|c}
	\caption{ RethinkDB secondary index and Query for Q13}
	\label{tbl:rethinkdb-index-query}\\
    {Index} & {Query}\\
	\hline
\begin{minipage}{.3\textwidth}
\begin{lstlisting}[language=JSON,basicstyle=\scriptsize]
    r.table("regions")
        .indexCreate("regions")
\end{lstlisting}
\end{minipage} &
\begin{minipage}{.5\textwidth}
\begin{lstlisting}[language=JSON,basicstyle=\scriptsize]
r.table("regions")
.getAll("australia",{index:"regions"})
    .map({  
       item:{  
          name:r.row("name"),
          description:r.row("description")
       }
    })
\end{lstlisting}
\end{minipage}
\end{longtable}
Some of the importain queries in RethinkDB explained here.


For query Q1, the document can be retrieved by the primary key \textit{id} using \textit{get()} function.  ReQl has more features for arrays and objects compare to  other NoSQL databases that helped in Q2 and Q3. Due to native support for joins between table, join queries are more flexible in RethinkDB. An index \textit{buyer\_person} is created in field "buyer.person" of \textit{closed\_auctions} table for Q8. Similarly, the Q9 can be improved by creating indexes in the tables \textit{regions} and \textit{closed\_auctions}. Other queries like Q13 and Q19 is also evaluated using indexes. For substring search in Q14, the value of  field \textit{description} is converted into text and then it is matched for the sub-string. All the queries and indexes are used for RethinkDB is given in ~\ref{rethinkdb-query-list}.
