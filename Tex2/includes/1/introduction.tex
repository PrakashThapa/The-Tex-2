\section{Introduction}
Data is growing very rapidly recent years. The huge amount of data collected from different field like business, medical and scientific research centers and the web etc. produces data that do not follow the specified format. Such data  is the main causes of this growth. Traditional data management systems are designed to store and process the data that To handle such unstructured data two new database system XML DBMS and NoSQL database system came to exists and success in their way. In this thesis we will study the similarities of these two new DBMS and the way they manage data. 

\section{Motivation}
		\label{motivation}
			The huge amount of data collected from different field like business, medical and scientific research in the present digital world has increased the complexity in data storage and querying. The traditional data management tools such as relational database management systems(RDBMS) are struggling to handle the data effectively  because of increment in  volume and the nature of data. The   pre-design rigid schema structure of RDBMS  has increased the complexities to process such variety of data.  has become a global phenomena which added more difficulties on RDBMS. RBDBMS are not able to store and process big data (that commonly has unstructured data) effectively and are not very efficient to make transactions and join operations~\citep{abramova2013nosql}.
			
			To overcome the drawback of RDBMS, NoSQL and  XML Databases has been used to store and process unstructured and semi-structured data(data that do not follow specific data format varies and can grow in real time). Nowadays, these two database has been used as an alternative of RDBMS in the data management system due to its efficiency in working with varieties of data in large volume, dynamic schema and scalability.
			%introduction to problem
			
			There are some research analysis between NoSQL and RDBMS. \cite{nance2013nosql} examine the pros and cons of both NoSQL and RDBMS (what are pros and cons? you need to write it down), \cite{cattell2011scalable} analyze similarities between scalable SQL and NoSQL databases, \cite{hadjigeorgiou2013rdbms}  compare the performance between MySQL cluster as RDBMS and Mongodb as NoSQL Database.  There are also some research related to RDBMS with XML or XML Database(~\citet{jiang2002xparent}, \citet{shanmugasundaram1999relational}). But there are not much more research in the field of NoSQL and XML database together. So this thesis focus on these two new database systems. Migration of Data from XML database to NoSQL as well as performance of both system based on standard query will be examined.	
	\section{Contribution}
		%\input{includes/1/contribution}
		The main contribution of this thesis is that it provides the necessary techniques and algorithms for migrating data from an XML database to NoSQL databases.  This thesis analyze the efficient of XML and NoSQL method in huge as well as in small database.  Moreover, it explains the different database like MongoDB, Couchbase and RethinkDB as NoSQL databases and BaseX as XML Database. To approach this challenge, (? Not clear about the challenge) it is necessary to understand the general architecture and data model of each databases and queried method. The performance comparison of XML and NoSQL method is also carried out based on XMARK dataset and its 20 standard queries~\citep{xmark/original}. These 20 queries of XMARK project is translated to each of NoSQL databases.
	\section{Overview }
		This master thesis is organize as follow: In chapter~\ref{nosql-xml-database}, we will introduce the two new databases systems. Chapter~\ref{semi-structure-data} defines the techniques and necessary algorithms to convert XML  to JSON. In same section, the XMARK dataset will be introduced and procedure to store in NoSQL databases. In Chapter~\ref{ch:benchmarking}, the spotlight is on the performance tests and comparative analysis of each of the NoSQL databases with BaseX, an native XML database, based on the XMark benchmarking project.		
	