\begin{figure}[hbtp]
\centering
\begin{lstlisting}[language=JSON,basicstyle =\scriptsize]

	{
		 <@\textbf{"id": "person0"}@>,
		"name": "Kasidit Treweek",
		"emailaddress": "mailto:Treweek@cohera.com",
		"phone": "+0 (645) 43954155",
		"homepage": "http://www.cohera.com/~Treweek",
		"creditcard": "9941 9701 2489 4716",
		"profile": {
			"income": 20186.59,
			"interest": [
			    { "category": "category251" },
				{"category": "category341"}
			],
			"education": "Graduate School",
			"business": "No"
		}
	}
\end{lstlisting}
\caption{RethinkDB document \textit{person0} in \textit{people} table}
\label{code:rethindb-person0}
\end{figure}
RethinkDB stores the documents inside a table which is identical to the collection in MongoDB. 
The documents are grouped according to their \textit{doctype} and store in a table.
Each \textit{doctype} of \ref{xmark-nosql} is represented as an individual table. 
The tables \textit{regions}, \textit{people}, \textit{open\_auctions}, \textit{closed\_auctions}, \textit{catgraph} and \textit{categories} contains the respective documents as of \textit{doctype}. Hence the attribute \textit{doctype} is removed from all documents.  \textit{id} is the primary key and any document without \textit{id} field is automatically added as an identifier during the time of insertion. Figure~\ref{code:rethindb-person0} shows a document with id \textit{person0} in \textit{people} table.
