A synthetic XMARK dataset consists of a huge record in tree structure~\citep{xmark/VIST}. As mentioned in Section~\ref{xmark-dataset}, each subtree, \textit{regions}, \textit{people}, \textit{open\_auctions}, \textit{closed\_auctions}, \textit{catgraph} and \textit{categories} contain large numbers of instances that are indexed during database creation. At first, in most NoSQL database, the dataset cannot be a huge block but in fragmented form with each instances having it's own index structure. Besides this, NoSQL databases limits the size of a single document. For example, MongoDB has a limitation of 16 MB per document, the maximum size of documents allowed in RethinkDB is 64 MB and Couchbase Server can have the value of a key up to 20 MB. The data model of NoSQL does not match the single-instance model of XML database.
\par 
To model XMark dataset into NoSQL, we have broken down the tree structure of XMark into set of sub-structures without losing the overall data. Each NoSQL database has their own data model, hence it is required to define a model for each of those databases separately. The generalized concept of  XMark data into NoSQL databases is explained here but it might slightly differ from one another. 

All sub-trees \textit{regions}, \textit{people}, \textit{open\_auctions}, \textit{closed\_auctions}, \textit{catgraph} and \textit{categories} are the basic units for the document fragmentation. Each of these sub-trees stores entities like \textit{item}, \textit{person}, \textit{open\_auction}, \textit{closed\_auction} and \textit{category}, respectively, as mentioned in Section~\ref{xmark-dataset}. These entities represent the documents in NoSQL databases. In each document, one special field \textit{doctype} is added to represent the name of the parent sub-tree. For example, in case of the \textit{people} sub-tree, the value of \textit{doctype} is \textit{people}. This key/value set will be the part of a document as  given in Table~\ref{tbl:xmark-xml-json}(b). The \textit{doctype} has altogether six distinct values : \textit{categories}, \textit{catgraph}, \textit{people}, \textit{open\_auctions} and \textit{closed\_auctions}. There is an exceptional case for \textit{item} entities. It has \textit{regions} as grandparent and the names of different regions like \textit{asia}, \textit{europe}, \textit{australia}, \textit{namerica}, \textit{samerica} etc. as the parent.  The \textit{doctype} for \textit{item} documents will be \textit{regions} as other. To represent the name of regions like \textit{asia}, \textit{europe}, etc.,  one field with key and value \textit{regions} is added in each document. 
Table~\ref{tbl:xmark-item-type} illustrates the extra attribute added in each of document.
\begin{longtable}{c|c}
	\caption{ Extra attribute of a document in NoSQL}
	\label{tbl:xmark-item-type}\\
    {\textit{item}  with \textit{asia} region} & {\textit{person0} }\\
	\hline
\begin{minipage}{.4\textwidth}
\begin{lstlisting}[language=JSON,basicstyle=\ttfamily\footnotesize]
{
	"doctype":"regions",
	"regions":"asia"
}
\end{lstlisting}
\end{minipage}
& 
\begin{minipage}{.4\textwidth}
\begin{lstlisting}[language=JSON,basicstyle=\ttfamily\footnotesize]
{
	"doctype":"people"
}
\end{lstlisting}
\end{minipage} 
\end{longtable}
\newpage
A sample document of NoSQL database along with respective XMark document is illustrated in  Table~\ref{tbl:xmark-xml-json}. The conversion from XML to JSON is carried out using algorithms of Section~\ref{xml-to-json-migration} with one extra attribute "doctype" to represent the parent of a document. If an element in XML has siblings with the same name, they are represented as an array in NoSQL document which is already mentioned in algorithm~\ref{algorithm-JSONXML}. As it can be seen, the \textit{person} element of XMark is itself a document in NoSQL, so it is not necessary to represent this attribute.  
\begin{longtable}{c|c}
	\caption{Example: XMARK data with id \textit{person0} \\in XML and JSON format }
	\label{tbl:xmark-xml-json}\\
	{\textit{person0}} in XML(a) & {\textit{person0}} in JSON for a NoSQL database(b)\\
	\hline
	\begin{minipage}{.4\textwidth}
\centering		
\begin{lstlisting}[language=XML,basicstyle = \tiny,label=code:xml-nosql-person0]
<people>
    <person id="person0">
       <name>Kasidit Treweek</name>
       <emailaddress>mailto:Treweek@cohera.com
            </emailaddress>
       <phone>+0 (645) 43954155</phone>
       <homepage>
            http://www.cohera.com/~Treweek
        </homepage>
       <creditcard>9941 9701 2489 4716
            </creditcard>
       <profile income="20186.59">
          <<@\textcolor{red}{interest category="category251" }@>/>
          <<@\textcolor{red}{interest category="category341" }@>/>
          <education>Graduate School
                </education>
          <business>No</business>
       </profile>
    </person>
</people>
\end{lstlisting}	
	\end{minipage} &
	\begin{minipage}{.49\textwidth}
		\centering
		\begin{lstlisting}[language=JSON, basicstyle =\tiny, label=code:json-nosql-person0, numberstyle=\tiny]
{
	"id": "person0",
	<@\textit{"doctype": "people",}@>
	"name": "Kasidit Treweek",
	"emailaddress": "mailto:Treweek@cohera.com",
	"phone": "+0 (645) 43954155",
	"homepage": "http://www.cohera.com/~Treweek",
	"creditcard": "9941 9701 2489 4716",
	"profile": {
		"income": 20186.59,
		<@\textcolor{red}{
		"interest": [\{
			"category": "category251"
		\},\{
			"category": "category341"
		\}]}@>,
		"education": "Graduate School",
		"business": "No"
	}
}
		\end{lstlisting}
	\end{minipage}\\
\end{longtable}


\newbox\mongodbXmarkChart
\begin{lrbox}{\mongodbXmarkChart}
\begin{tikzpicture}[grow'=right,level distance=1.25in,sibling distance=.25in, font=\normalsize]
\tikzset{edge from parent/.style= 
            {thick, draw, edge from parent fork right},
         every tree node/.style=
            {draw,minimum width=1in,text width=1in,align=center}}
\Tree 
    [. Database 
        [.{regions}
            [.{... } ]
        ]
        [.people
            [.{person0 } ]
        ] 
        [.{open\_auctions}
            [.{... } ]
        ]
        [.{closed\_auctions}
            [.{... } ]
        ]
        [.{catgraph}
            [.{... } ]
        ]
        [.{catgraph}
            [.{... } ]
        ]
    ]
\end{tikzpicture}
\end{lrbox}
\newbox\mongodbXmarkDocument
\begin{lrbox}{\mongodbXmarkDocument}
\begin{lstlisting}[language=JSON, basicstyle=\ttfamily\footnotesize]
    {
    	<@\textit{"\_id": "person0",}@>
    	"name": "Kasidit Treweek",
    	"emailaddress": "mailto:Treweek@cohera.com",
    	"phone": "+0 (645) 43954155",
    	"homepage": "http://www.cohera.com/~Treweek",
    	"creditcard": "9941 9701 2489 4716",
    	"profile": {
    		"income": 20186.59,
    		"interest": [
    			{"category": "category251"},
    			{"category": "category341"}
    			],
    		"education": "Graduate School",
    		"business": "No"
    	}
    }
\end{lstlisting}
\end{lrbox}




\subsubsection{XMARK dataset in MongoDB} \label{xmark-mongodb}
%In MongoDB, collections consist of a group of documents with similar structure. Therefore, the data modeling concept of Section~\ref{xmark-nosql} has to be modified marginally. The documents are grouped with their \textit{doctype} from Section~\ref{xmark-nosql}. Each \textit{doctype} represents a collection, there are 6 collections. The field \textit{doctype} is already represented as a collection, it is removed from all documents.  
For \textit{item} entity,  field \textit{regions} does not change. The \textit{\_id} is the primary index of a document in MongoDB, the identifier attribute of the XMark data \textit{id} is renamed to \textit{\_id} for default indexing.  \textit{closed\_auctions} and \textit{catgraph} do not have an identifier attribute \textit{id}, therefore, the system will automatically generate the \textit{\_id} in these collections. The XMark dataset in MongoDB is represented in Figure~\ref{xmark-mongodb-tree}.
A typical example of MongoDB document for a person with identifier \textit{person0} is given in Figure~\ref{code:mongodb-person0}.	

\newbox\mongodbXmarkDocument
\begin{lrbox}{\mongodbXmarkDocument}
\begin{lstlisting}[language=JSON, basicstyle =\scriptsize]
    {
    	<@\textbf{"\_id": "person0",}@>
    	"name": "Kasidit Treweek",
    	"emailaddress": "mailto:Treweek@cohera.com",
    	"phone": "+0 (645) 43954155",
    	"homepage": "http://www.cohera.com/~Treweek",
    	"creditcard": "9941 9701 2489 4716",
    	"profile": {
    		"income": 20186.59,
    		"interest": [
    			{"category": "category251"},
    			{"category": "category341"}
    			],
    		"education": "Graduate School",
    		"business": "No"
    	}
    }
\end{lstlisting}
\end{lrbox}


\newbox\mongodbXmarkChart
\begin{lrbox}{\mongodbXmarkChart}
\begin{tikzpicture}[grow'=right,level distance=1.25in,sibling distance=.25in, font=\scriptsize]
\tikzset{edge from parent/.style= 
            {thick, draw, edge from parent fork right},
         every tree node/.style=
            {draw,minimum width=1in,text width=1in,align=center}}
\Tree 
    [. Database 
        [.{regions}
            [.{... } ]
        ]
        [.people
            [.{person0 } ]
        ] 
        [.{open\_auctions}
            [.{... } ]
        ]
        [.{closed\_auctions}
            [.{... } ]
        ]
        [.{catgraph}
            [.{... } ]
        ]
        [.{catgraph}
            [.{... } ]
        ]
    ]
    
\end{tikzpicture}
\end{lrbox}

\begin{figure}[hbtp]
\centering
\subfloat[Database, collections and documents in MongoDB] {
    \usebox\mongodbXmarkChart
    \label{xmark-mongodb-tree}
}
\\
\centering
\subfloat[{MongoDB document of XMark data in \textit{people} collection} ] {
        \usebox\mongodbXmarkDocument
        \label{code:mongodb-person0}
}

\caption{XMark data in MongoDB}
\label{xmark-mongodb-figure}
\end{figure}

In MongoDB, collections consist of a group of documents with similar structure. Therefore, the data modeling concept of Section~\ref{xmark-nosql} has to be modified marginally. The documents are grouped with their \textit{doctype} from Section~\ref{xmark-nosql}. Each \textit{doctype} represents a collection, there are 6 collections. The field \textit{doctype} is already represented as a collection, it is removed from all documents.  
For \textit{item} entity,  field \textit{regions} does not change. The \textit{\_id} is the primary index of a document in MongoDB, the identifier attribute of the XMark data \textit{id} is renamed to \textit{\_id} for default indexing.  \textit{closed\_auctions} and \textit{catgraph} do not have an identifier attribute \textit{id}, therefore, the system will automatically generate the \textit{\_id} in these collections. The XMark dataset in MongoDB is represented in Figure~\ref{xmark-mongodb-tree}.
A typical example of MongoDB document for a person with identifier \textit{person0} is given in Figure~\ref{code:mongodb-person0}.	



\begin{figure}[hbtp]
\centering
\subfloat[Database, collections and documents in MongoDB] {
    \usebox\mongodbXmarkChart
    \label{xmark-mongodb-tree}
}
\\
\centering
\subfloat[{MongoDB document of XMark data in \textit{people} collection} ] {
        \usebox\mongodbXmarkDocument
        \label{code:mongodb-person0}
}

\caption{XMark data in MongoDB}
\label{xmark-mongodb-figure}
\end{figure}
				
\subsubsection{XMARK dataset in Couchbase} \label{xmark-couchbase}


%
Couchbase does not have concept of grouping of documents like \textit{collections} in MongoDB  or \textit{tables} of RethinkDB. Therefore, the data model of Section ~\ref{xmark-nosql} is applied without modification. All the documents are stored in a single bucket independent of each other with \textit{id} as a document key. An \textit{id} will be manually generated for the documents without identifier.
 An example of Couchbase document is illustrated in Figure~\ref{code:couchbase-item0}
%\subsubsubsection{Queries}
%write full code

\begin{figure}
\begin{lstlisting}[language=JSON,  basicstyle =\scriptsize]
{
	"id":  "item1000",
	"doctype":  "regions",
	"regions":  "africa",
	"name":  "duteous nine eighteen" ,
	"payment":  "Creditcard" ,
	"quantity": 1 ,
	"shipping":  "Will ship internationally, See description for charges" ,
	"incategory": [
		{
		"category":  "category0"
		}
	] ,
	"mailbox":[],
	"description":{ }
}
\end{lstlisting} 
\caption{Couchbase pseudo-document of the XMark data for item with id \textit{item1000}}
\label{code:couchbase-item0}
\end{figure}

Couchbase does not have the concept of grouping documents as \textit{collections} in MongoDB  or \textit{tables} in RethinkDB. 
Therefore, the data model of Section ~\ref{xmark-nosql} is applied without modification.
 All the documents of an XMark instance  are stored in a single bucket with identifier attribute \textit{id} as a document key. An \textit{id} will be manually generated for the documents without identifier.
An example of Couchbase document is illustrated in Figure~\ref{code:couchbase-item0}. The grahical representation of overall dataset in MongoDB is shown in ~\ref{xmark-cb-figure}. The identifier attribute \textit{id}  of XMark data is the document key in Couchbase. The \textit{doctype} with  value \textit{regions} represents it as an \textit{item} and finally the \textit{regions} attribute serves to represent which regions this \textit{item} associates with.

\newbox\cbXmarkChart
\begin{lrbox}{\cbXmarkChart}
\begin{tikzpicture}[grow'=right,level distance=1.25in,sibling distance=.25in, font=\normalsize]
\tikzset{edge from parent/.style= 
            {thick, draw, edge from parent fork right},
         every tree node/.style=
            {draw,minimum width=1in,text width=1in,align=center}}
\Tree 
    [. Bucket 
        [.{item0 } ]
        [.{... } ]
        [.{item1000 } ]
        [.{person0 } ]
        [.{... } ]
        [.{category0 } ]
        [.{... } ]
    ]
    
\end{tikzpicture}
\end{lrbox}
\newbox\cbXmarkDocument
\begin{lrbox}{\cbXmarkDocument}
\begin{lstlisting}[language=JSON,  basicstyle=\ttfamily\footnotesize]
                {
                	<@\textbf{"id": "item1000"}@>,
                	<@\textcolor{red}{"doctype":  "regions"}@>,
                	<@\textit{"regions":  "africa"}@>,
                	"name":  "duteous nine eighteen" ,
                	"payment":  "Creditcard" ,
                	"quantity": 1 ,
                	"shipping":  "Will ship internationally, See description for charges" ,
                	"incategory": [{
                		"category":  "category0"
                		}] ,
                	"mailbox":[],
                	"description":{}
                }
\end{lstlisting} 
\end{lrbox}


\begin{figure}[h]
\centering
\subfloat[Bucket and documents in Couchbase] {
    \usebox\cbXmarkChart
    \label{xmark-cb-tree}
}
\\
\centering
\subfloat[Couchbase document of the XMark data for item with id \textit{item1000} ] {
        \usebox\cbXmarkDocument
        \label{code:couchbase-item0}
}

\caption{XMark data in Couchbase}
\label{xmark-cb-figure}
\end{figure}



\subsubsection{XMARK dataset in Rethinkdb} \label{xmark-rethinkdb}
%RethinkDB stores the documents inside a table which is identical to the collection in MongoDB. 
The documents are grouped according to their \textit{doctype} and store in a table.
Each \textit{doctype} of \ref{xmark-nosql} is represented as an individual table. 
The tables \textit{regions}, \textit{people}, \textit{open\_auctions}, \textit{closed\_auctions}, \textit{catgraph} and \textit{categories} contains the respective documents as of \textit{doctype}. Hence the attribute \textit{doctype} is removed from all documents.  \textit{id} is the primary key and any document without \textit{id} field is automatically added as an identifier during the time of insertion. Figure~\ref{code:rethindb-person0} shows a document with id \textit{person0} in \textit{people} table.


\newbox\rethinkdbXmarkDocument
\begin{lrbox}{\rethinkdbXmarkDocument}
\begin{lstlisting}[language=JSON,basicstyle =\scriptsize]

	{
		 <@\textbf{"id": "person0"}@>,
		"name": "Kasidit Treweek",
		"emailaddress": "mailto:Treweek@cohera.com",
		"phone": "+0 (645) 43954155",
		"homepage": "http://www.cohera.com/~Treweek",
		"creditcard": "9941 9701 2489 4716",
		"profile": {
			"income": 20186.59,
			"interest": [
			    { "category": "category251" },
				{"category": "category341"}
			],
			"education": "Graduate School",
			"business": "No"
		}
	}
\end{lstlisting}
\end{lrbox}


\newbox\rethinkdbXmarkChart
\begin{lrbox}{\rethinkdbXmarkChart}
\begin{tikzpicture}[grow'=right,level distance=1.25in,sibling distance=.25in, font=\scriptsize]
\tikzset{edge from parent/.style= 
            {thick, draw, edge from parent fork right},
         every tree node/.style=
            {draw,minimum width=1in,text width=1in,align=center}}
\Tree 
    [. Database 
        [.{regions}
            [.{... } ]
        ]
        [.people
            [.{person0 } ]
        ] 
        [.{open\_auctions}
            [.{... } ]
        ]
        [.{closed\_auctions}
            [.{... } ]
        ]
        [.{catgraph}
            [.{... } ]
        ]
        [.{catgraph}
            [.{... } ]
        ]
    ]
    
\end{tikzpicture}
\end{lrbox}

\begin{figure}[hhtp]
\centering
\subfloat[Database, tables and documents in RethinkDB] {
    \usebox\rethinkdbXmarkChart
    \label{xmark-rethinkdb-tree}
}
\\
\centering
\subfloat[RethinkDB document \textit{person0} in \textit{people} table ] {
        \usebox\rethinkdbXmarkDocument
        \label{code:rethindb-person0}
}

\caption{XMark data in RethinkDB}
\label{xmark-rethinkdb-figure}
\end{figure}

RethinkDB stores the documents inside a table which is identical to the collection in MongoDB. 
The documents are grouped according to their \textit{doctype} and store in a table.
Each \textit{doctype} of \ref{xmark-nosql} is represented as an individual table. 
The tables \textit{regions}, \textit{people}, \textit{open\_auctions}, \textit{closed\_auctions}, \textit{catgraph} and \textit{categories} contains the respective documents as \textit{doctype}. Therefore, the attribute \textit{doctype} is removed from all documents.  \textit{id} is the primary key and any document without \textit{id} field is automatically added as an identifier during the time of insertion. Figure~\ref{code:rethindb-person0} shows a document with id \textit{person0} in \textit{people} table.


\newbox\rethinkdbXmarkChart
\begin{lrbox}{\rethinkdbXmarkChart}
\begin{tikzpicture}[grow'=right,level distance=1.25in,sibling distance=.25in, font=\scriptsize]
\tikzset{edge from parent/.style= 
            {thick, draw, edge from parent fork right},
         every tree node/.style=
            {draw,minimum width=1in,text width=1in,align=center}}
\Tree 
    [. Database 
        [.{regions}
            [.{... } ]
        ]
        [.people
            [.{person0 } ]
        ] 
        [.{open\_auctions}
            [.{... } ]
        ]
        [.{closed\_auctions}
            [.{... } ]
        ]
        [.{catgraph}
            [.{... } ]
        ]
        [.{categories}
            [.{... } ]
        ]
    ]
    
\end{tikzpicture}
\end{lrbox}



\newbox\rethinkdbXmarkDocument
\begin{lrbox}{\rethinkdbXmarkDocument}
\begin{lstlisting}[language=JSON,basicstyle=\ttfamily\footnotesize]
        	{
        		 <@\it{"id": "person0"}@>,
        		"name": "Kasidit Treweek",
        		"emailaddress": "mailto:Treweek@cohera.com",
        		"phone": "+0 (645) 43954155",
        		"homepage": "http://www.cohera.com/~Treweek",
        		"creditcard": "9941 9701 2489 4716",
        		"profile": {
        			"income": 20186.59,
        			"interest": [
        			    { "category": "category251" },
        				{"category": "category341"}
        			],
        			"education": "Graduate School",
        			"business": "No"
        		}
        	}
\end{lstlisting}
\end{lrbox}

\begin{figure}[H]
\centering
\subfloat[Database, tables and documents in RethinkDB] {
    \usebox\rethinkdbXmarkChart
    \label{xmark-rethinkdb-tree}
}
\\
\centering
\subfloat[RethinkDB document \textit{person0} in \textit{people} table ] {
        \usebox\rethinkdbXmarkDocument
        \label{code:rethindb-person0}
}

\caption{XMark data in RethinkDB}
\label{xmark-rethinkdb-figure}
\end{figure}
