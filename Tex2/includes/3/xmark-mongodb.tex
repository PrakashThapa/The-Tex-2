MongoDB collection consists group of documents similar structure. Therefore, the data modeling concept of \ref{xmark-nosql} has to be modified marginally for MongoDB. Each \textit{doctype} from Section~\ref{xmark-nosql} represents a collection. For \textit{item} entity,  attribute \textit{regions} contains the name of the region of each item. The \textit{\_id} is the primary index of a document, the identifier attribute of the XMark data \textit{id} is renamed to \textit{\_id} for default indexing.  \texttt{closed\_auctions} and \texttt{catgraph} do not have an identifier \textit{id}, therefore, system will automatically generate the \texttt{\_id} in these collections. A typical example of MongoDB document for person with identifier \textit{person0} is given in Figure~\ref{code:mongodb-person0}.	
\begin{figure}
\begin{lstlisting}[language=JSON, basicstyle =\scriptsize]
	{
		"_id": "person0",
		"name": "Kasidit Treweek",
		"emailaddress": "mailto:Treweek@cohera.com",
		"phone": "+0 (645) 43954155",
		"homepage": "http://www.cohera.com/~Treweek",
		"creditcard": "9941 9701 2489 4716",
		"profile": {
			"income": 20186.59,
			"interest": [{
				"category": "category251"
			}],
			"education": "Graduate School",
			"business": "No"
		}
	}
\end{lstlisting}
\caption{MongoDB's document representation of XMARk data}
\label{code:mongodb-person0}
\end{figure}
