RethinkDB is an open-source distributed document-oriented database system to store JSON documents. One of the unique feature of RethinkDB is the changefeed by the server to the client. 
Instead of a client request, the changes in database is streamed to the client application in realtime. In case of multi-users environment, any database  update is automatically notified.
\par
ReQL is the query language  of RethinkDB and it is based on three main principles:
 \begin{itemize}
 \item  It is  embedded  as programming language. Queries are constructed by making function call. 
 \item ReQL queries are chainable that can be passed as pipeline from one stage to another. Complex operation can be performed using series of simple queries together by using dot operators (\.) . 
 \item All the queries are executed in server without any intermediate network round trip between the server and clients.
 \end{itemize}
  
\subsubsection{Data Model}
There are two ways to model relationships between the documents: 
\begin{itemize}
	\item \textbf{Embedded arrays:} In this method, the related sub-documents are inserted with a specific key inside a document as in MongoDB ~\ref{mongo:embedded}. The advantages of using embedded arrays are:
		\begin{itemize}
			\item The Queries tend to be simpler. 
			\item If a dataset  does not fit into RAM, then the data is loaded  from the disk and it is faster compared to tables. 
		\end{itemize}
		Disadvantages of embedded arrays are: 
		\begin{itemize}
			\item Before any operation, data should be loaded into memory. In case of any updates, the document will rewrite the full array to the disk.
		\end{itemize}
		
	\item 
	\textbf{Multiple Table Approach:} In this approach, documents with similar structures are stored in multiple tables similar to collection in MongoDB. These tables are connected by reference key. Unlike embedded arrays, operation of a table does not required to load data from reference table in memory.
\end{itemize}

\subsubsection{Query Model}
RethinkDB's query language ReQL is embedded as programming language and JavaScript expressions can be used anywhere as a part of the query. The anonymous function, also known as lamda function, is a part of the query language that gives more flexibility for retrieving data. All the ReQL queries are chainable therefore, the dot \{.\} operator at any point of query can be extended to multiple levels as shown in Code~\ref{reql-chainable}.
	\begin{lstlisting}[language=JSON,caption=Chainable Query in ReQL, label=reql-chainable, xleftmargin=-40pt][h]
		r.table("users").run(conn)
		r.table("users").pluck("last_name").run(conn)
		r.table("users").pluck("last_name").distinct().run(conn)
		r.table("users").pluck("last_name").distinct().count().run(conn)
	\end{lstlisting} 
The queries are built up on client side and send to the server when the \textit{run()} is called. Then the server automatically parallelized the queries into different nodes. whenever possible, the query execution is splitted into different cluster and data-centers.
\par
Unlike other NoSQL databases, RethinkDB supports join queries between the tables in one-to-many or many-to-many relationships in distributed manner. 
%start here from bechmarking section
%there should be some changes
\subsubsection{Indexing}
	RethinkDB uses B-trees to store indexes. When tables are created, there is an option to specify the attribute as a primary key. \textit{id} behaves as a primary key if the attribute is not specified and it is used to index the document. If there is no  primary key, a random unique string is automatically generated to index the document. Beside the primary index, RethinkDB supports simple, compound, secondary, geospatial indexes. Beside that any arbitrary expression can be used to create index from any type of user defined expressions like lamda functions. Every query including update operations uses only one index.
	 Only \textit{getAll()} \textit{between()}, \textit{eqJoin()} and \textit{orderBy} functions can use secondary index.