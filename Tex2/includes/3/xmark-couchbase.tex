Couchbase does not have the concept of grouping documents like \textit{collections} in MongoDB  or \textit{tables} in RethinkDB. 
Therefore, the data model of Section ~\ref{xmark-nosql} is applied without modification.
 All the documents of an XMark instance  are stored in a single bucket with identifier attribute \textit{id} as a document key. An \textit{id} will be manually generated for the documents without identifier.
An example of Couchbase document is illustrated in Figure~\ref{code:couchbase-item0}. The \textit{id} attribute is identifier of XMark data and document key in Couchbase, the \textit{doctype} with  value \textit{regions} represents it is and item and finally the \textit{regions} attributes is to represent the in which regions does this attributes belongs to.\todo{english}
\begin{figure}[hbt]
\begin{lstlisting}[language=JSON,  basicstyle =\scriptsize]
    {
    	<@\textbf{"id": "item1000"}@>,
    	<@\textcolor{red}{"doctype":  "regions"}@>,
    	<@\textit{"regions":  "africa"}@>,
    	"name":  "duteous nine eighteen" ,
    	"payment":  "Creditcard" ,
    	"quantity": 1 ,
    	"shipping":  "Will ship internationally, See description for charges" ,
    	"incategory": [{
    		"category":  "category0"
    		}] ,
    	"mailbox":[],
    	"description":{}
    }
\end{lstlisting} 
\caption{Couchbase document of the XMark data for item with id \textit{item1000}}
\label{code:couchbase-item0}
\end{figure}
