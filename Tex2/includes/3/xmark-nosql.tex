A synthetic XMARK dataset consists of a huge record in tree structure~\citep{xmark/VIST}. As mentioned in Section~\ref{xmark-dataset}, each subtree, \textit{regions}, \textit{people}, \textit{open\_auctions}, \textit{closed\_auctions}, \textit{catgraph} and \textit{categories} contain large numbers of instances that are indexed during database creation. At first, in most NoSQL database, the dataset cannot be a huge block but in fragmented form with each instances having it's own index structure. Besides this, NoSQL databases limits the size of a single document. For example, MongoDB has a limitation of 16 MB per document, the maximum size of documents allowed in RethinkDB is 64 MB and Couchbase Server can have value of a key upto 20 MB. The data model of NoSQL does not match single instance model of XML database.
\par 
To model XMark dataset into NoSQL, we have broken down the tree structure of XMark  into set of sub-structure without losing the overall data. Each NoSQL database has their own data model, hence it is required to define a model for each of those databases separately.

The generalized concept of \underline {XMark data into NoSQL databases} might slightly differ from one another. 

\textbf{All sub-trees \textit{regions}, \textit{people}, \textit{open\_auctions}, \textit{closed\_auctions}, \textit{catgraph} and \textit{categories} are the basic unit for the document fragmentation}. Each of these sub-trees stores entities \textit{item}, \textit{person}, \textit{open\_auction}, \textit{closed\_auction} and \textit{category} respectively as mentioned in ~\ref{xmark-dataset}. These entities represent the documents in NoSQL database. In each document, one special field \textit{doctype} is added to represent the name of parent sub-tree. For example, in case of \textit{people} sub-tree, the value of \textit{doctype} is \textit{people}. This key-value set will be the part of a document as  given in Table~\ref{tbl:xmark-xml-json}(b). Therefore, \textit{doctype} has all-together six distinct values : \textit{categories}, \textit{catgraph}, \textit{people}, \textit{open\_auctions} and \textit{closed\_auctions}. There is an exceptional case for \textit{item} entities. It has \textit{regions} as grandparent and name of different regions like \textit{asia}, \textit{europe}, \textit{australia}, \textit{namerica}, \textit{samerica} etc. as the parent.  The \textit{doctype} for \textit{item} document will be \textit{regions} as other. To represent the name of regions like \textit{asia}, \textit{europe}, etc.,  one field with key \textit{regions} is added in each document. 
Table~\ref{tbl:xmark-item-type} illustrate the extra attribute added in each of document.

\begin{longtable}{c|c}
	\caption{ Extra attribute of a document in NoSQL}
	\label{tbl:xmark-item-type}\\
    {for \textit{person} and all other except \textit{item} } & {for \textit{item} which has region name \textit{asia}}\\
	\hline
\begin{minipage}{.4\textwidth}
\begin{lstlisting}[language=JSON]
{
	"doctype":"people"
}
\end{lstlisting}
\end{minipage} &
\begin{minipage}{.4\textwidth}
\begin{lstlisting}[language=JSON]
{
	"doctype":"regions",
	"regions":"asia"
}
\end{lstlisting}
\end{minipage}
\end{longtable}

The mapping of XML to JSON is shown in Table~\ref{tbl:xmark-xml-json}. The conversion from XML to JSON is carried out using  algorithms of Section~\ref{xml-to-json-migration} with one extra attribute "doctype" to represent the parent of a document.
\begin{longtable}{c|c}
	\caption{Example: XMARK data with id \textit{person1023} in XML and JSON format }
	\label{tbl:xmark-xml-json}\\
	{\textit{person1023}} in XML(a) & {\textit{person0}} in JSON for a NoSQL database(b)\\
	\hline
	\begin{minipage}{.4\textwidth}
\centering		
\begin{lstlisting}[language=XML,basicstyle = \tiny,label=code:xml-nosql-person0]
<people>
    <person id="person1023">
       <name>Kasidit Treweek</name>
       <emailaddress>mailto:Treweek@cohera.com</emailaddress>
       <phone>+0 (645) 43954155</phone>
       <homepage>http://www.cohera.com/~Treweek</homepage>
       <creditcard>9941 9701 2489 4716</creditcard>
       <profile income="20186.59">
          <interest category="category251" />
          <education>Graduate School</education>
          <business>No</business>
       </profile>
    </person>
</people>
\end{lstlisting}	
	\end{minipage} &
	\begin{minipage}{.55\textwidth}
		\centering
		\begin{lstlisting}[language=JSON, basicstyle =\tiny, label=code:json-nosql-person0]
{
	"id": "person0",
	"doctype": "people",
	"name": "Kasidit Treweek",
	"emailaddress": "mailto:Treweek@cohera.com",
	"phone": "+0 (645) 43954155",
	"homepage": "http://www.cohera.com/~Treweek",
	"creditcard": "9941 9701 2489 4716",
	"profile": {
		"income": 20186.59,
		"interest": [{
			"category": "category251"
		}],
		"education": "Graduate School",
		"business": "No"
	}
}
		\end{lstlisting}
	\end{minipage}\\
\end{longtable}

%\subsubsection{XMark data in JSON for NoSQL}


\begin{comment}
\iffalse\fi
\begin{minipage}{.5\textwidth}
	\begin{tikzpicture}[%
	grow via three points={one child at (0.5,-0.7) and
		two children at (0.5,-0.7) and (0.5,-1.4)},
	edge from parent path={(\tikzparentnode.south) |- (\tikzchildnode.west)}]
	\node {\{asfdasfd\}}
	child { node [defi] {\textit{Schema\_ID}}}
	child { node [json] {xs:attribute}
		child { node [defi] {\textit{Attribute\_ID}}}
		child { node [attribute] {@name}}
		child { node [attribute] {@type}}
		child { node [attribute] {@fixed}}
		child { node [attribute] {@default}}
	};
	\end{tikzpicture}
\end{minipage}

\end{comment}

