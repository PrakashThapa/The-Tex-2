
In MongoDB, collections consist of group of documents with similar structure. Therefore, the data modeling concept of Section~\ref{xmark-nosql} has to be modified marginally for MongoDB. The documents are grouped with their \textit{doctype} from Section~\ref{xmark-nosql}. Each \textit{doctype} represent a collection, there are all together 6 collections. The field \textit{doctype} is already represented as collections, it is removed from all documents.  
For \textit{item} entity,  field \textit{regions} does not change. The \textit{\_id} is the primary index of a document in MongoDB, the identifier attribute of the XMark data \textit{id} is renamed to \textit{\_id} for default indexing.  \textit{closed\_auctions} and \textit{catgraph} do not have an identifier attribute \textit{id}, therefore, system will automatically generate the \textit{\_id} in these collections.
A typical example of MongoDB document for person with identifier \textit{person0} is given in Figure~\ref{code:mongodb-person0}.	

\begin{figure}[hbt]
\begin{lstlisting}[language=JSON, basicstyle =\scriptsize]
    {
    	<@\textbf{"\_id": "person0",}@>
    	"name": "Kasidit Treweek",
    	"emailaddress": "mailto:Treweek@cohera.com",
    	"phone": "+0 (645) 43954155",
    	"homepage": "http://www.cohera.com/~Treweek",
    	"creditcard": "9941 9701 2489 4716",
    	"profile": {
    		"income": 20186.59,
    		"interest": [
    			{"category": "category251"},
    			{"category": "category341"}
    			],
    		"education": "Graduate School",
    		"business": "No"
    	}
    }
\end{lstlisting}
\caption{MongoDB document of XMark data in \textit{people} collection}
\label{code:mongodb-person0}
\end{figure}