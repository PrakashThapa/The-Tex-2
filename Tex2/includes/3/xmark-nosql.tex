A synthetic XMARK dataset consists of a huge record in tree structure~\citep{xmark/VIST}. As mentioned in Section~\ref{xmark-dataset}, each subtree, \textit{regions}, \textit{people}, \textit{open\_auctions}, \textit{closed\_auctions}, \textit{catgraph} and \textit{categories} contain large numbers of instances that are indexed during database creation. At first, in most NoSQL database, the dataset cannot be a huge block but in fragmented form with each instances having it's own index structure. Besides this, NoSQL databases limits the size of a single document. For example, MongoDB has a limitation of 16 MB per document, the maximum size of documents allowed in RethinkDB is 64 MB and Couchbase Server can have the value of a key up to 20 MB. The data model of NoSQL does not match the single-instance model of XML database.
\par 
To model XMark dataset into NoSQL, we have broken down the tree structure of XMark into set of sub-structures without losing the overall data. Each NoSQL database has their own data model, hence it is required to define a model for each of those databases separately. The generalized concept of  XMark data into NoSQL databases is explained here but it might slightly differ from one another. 

All sub-trees \textit{regions}, \textit{people}, \textit{open\_auctions}, \textit{closed\_auctions}, \textit{catgraph} and \textit{categories} are the basic units for the document fragmentation. Each of these sub-trees stores entities like \textit{item}, \textit{person}, \textit{open\_auction}, \textit{closed\_auction} and \textit{category}, respectively, as mentioned in Section~\ref{xmark-dataset}. These entities represent the documents in NoSQL databases. In each document, one special field \textit{doctype} is added to represent the name of the parent sub-tree. For example, in case of the \textit{people} sub-tree, the value of \textit{doctype} is \textit{people}. This key/value set will be the part of a document as  given in Table~\ref{tbl:xmark-xml-json}(b). The \textit{doctype} has altogether six distinct values : \textit{categories}, \textit{catgraph}, \textit{people}, \textit{open\_auctions} and \textit{closed\_auctions}. There is an exceptional case for \textit{item} entities. It has \textit{regions} as grandparent and the names of different regions like \textit{asia}, \textit{europe}, \textit{australia}, \textit{namerica}, \textit{samerica} etc. as the parent.  The \textit{doctype} for \textit{item} documents will be \textit{regions} as other. To represent the name of regions like \textit{asia}, \textit{europe}, etc.,  one field with key and value \textit{regions} is added in each document. 
Table~\ref{tbl:xmark-item-type} illustrates the extra attribute added in each of document.

\begin{longtable}{c|c}
	\caption{ Extra attribute of a document in NoSQL}
	\label{tbl:xmark-item-type}\\
    {for \textit{person} and all other entities except \textit{item} } & {for \textit{item} which has the region name \textit{asia}}\\
	\hline
\begin{minipage}{.4\textwidth}
\begin{lstlisting}[language=JSON]
{
	"doctype":"people"
}
\end{lstlisting}
\end{minipage} &
\begin{minipage}{.4\textwidth}
\begin{lstlisting}[language=JSON]
{
	"doctype":"regions",
	"regions":"asia"
}
\end{lstlisting}
\end{minipage}
\end{longtable}

A sample document of NoSQL database along with respective XMark document is illustrated in  Table~\ref{tbl:xmark-xml-json}. The conversion from XML to JSON is carried out using algorithms of Section~\ref{xml-to-json-migration} with one extra attribute "doctype" to represent the parent of a document. If an element in XML has siblings with the same name, they are represented as an array in NoSQL document which is already mentioned in algorithm~\ref{algorithm-JSONXML}. As it can be seen, the \textit{person} element of XMark is itself a document in NoSQL, so it is not necessary to represent this attribute.  

\begin{longtable}{c|c}
	\caption{Example: XMARK data with id \textit{person0} \\in XML and JSON format }
	\label{tbl:xmark-xml-json}\\
	{\textit{person0}} in XML(a) & {\textit{person0}} in JSON for a NoSQL database(b)\\
	\hline
	\begin{minipage}{.4\textwidth}
\centering		
\begin{lstlisting}[language=XML,basicstyle = \tiny,label=code:xml-nosql-person0]
<people>
    <person id="person0">
       <name>Kasidit Treweek</name>
       <emailaddress>mailto:Treweek@cohera.com
            </emailaddress>
       <phone>+0 (645) 43954155</phone>
       <homepage>
            http://www.cohera.com/~Treweek
        </homepage>
       <creditcard>9941 9701 2489 4716
            </creditcard>
       <profile income="20186.59">
          <<@\textcolor{red}{interest category="category251" }@>/>
          <<@\textcolor{red}{interest category="category341" }@>/>
          <education>Graduate School
                </education>
          <business>No</business>
       </profile>
    </person>
</people>
\end{lstlisting}	
	\end{minipage} &
	\begin{minipage}{.55\textwidth}
		\centering
		\begin{lstlisting}[language=JSON, basicstyle =\tiny, label=code:json-nosql-person0, numberstyle=\tiny]
{
	"id": "person0",
	<@\textit{"doctype": "people",}@>
	"name": "Kasidit Treweek",
	"emailaddress": "mailto:Treweek@cohera.com",
	"phone": "+0 (645) 43954155",
	"homepage": "http://www.cohera.com/~Treweek",
	"creditcard": "9941 9701 2489 4716",
	"profile": {
		"income": 20186.59,
		<@\textcolor{red}{
		"interest": [\{
			"category": "category251"
		\},\{
			"category": "category341"
		\}]}@>,
		"education": "Graduate School",
		"business": "No"
	}
}
		\end{lstlisting}
	\end{minipage}\\
\end{longtable}

\begin{comment}
\iffalse\fi
\begin{minipage}{.5\textwidth}
	\begin{tikzpicture}[%
	grow via three points={one child at (0.5,-0.7) and
		two children at (0.5,-0.7) and (0.5,-1.4)},
	edge from parent path={(\tikzparentnode.south) |- (\tikzchildnode.west)}]
	\node {\{asfdasfd\}}
	child { node [defi] {\textit{Schema\_ID}}}
	child { node [json] {xs:attribute}
		child { node [defi] {\textit{Attribute\_ID}}}
		child { node [attribute] {@name}}
		child { node [attribute] {@type}}
		child { node [attribute] {@fixed}}
		child { node [attribute] {@default}}
	};
	\end{tikzpicture}
\end{minipage}

\end{comment}

