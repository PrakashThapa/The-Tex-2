%\section{Introduction}
%Data is growing very rapidly recent years. The huge amount of data collected from different field like business, medical and scientific research centers and the web etc. produces data that do not follow the specified format. Such data  is the main causes of this growth. Traditional data management systems are designed to store and process the data that To handle such unstructured data two new database system XML DBMS and NoSQL database system came to exists and success in their way. In this thesis we will study the similarities of these two new DBMS and the way they manage data. 

\section{Motivation}
		\label{motivation}
		    As the digital world growing very fast, the massive amount of structured and unstructured data collected from various fields, is becoming more complex for storage, and querying.The traditional data management tools such as relational database management systems(RDBMS) are designed for structured data. That is, structured data has rows and columns which perfectly fit into relational database. On the other hand, semi-structured data contains different data types such as arrays and maps that are completely irrelevant to relational databases. Data stored in XML and NoSQL databases are considered to store semi-structured format.
		\par
		 Relational database works best with the well organized tables. The structure of the data is pre-defined by the layout of the table and the types of columns. The pre-design rigid schema structure of RDBMS has added the complexities to process such variety of data. It is possible to scale relational databases, but to scale beyond a point, it must be distributed across multiple servers. Joining the tables across distributed systems is very difficult. It is slow and inefficient, when dealing with complex data structures that does not fit into a table. Relational databases offer a big set of features, that often users do not need.
	\par
	To overcome the limitations of RDBMS, NoSQL and XML databases are used to store and process unstructured and semi-structured data. These databases supports horizontal scalability hence, the data is distributed in different nodes. As relational databases support ACID properties, there are several restrains to perform on every piece of the data which makes the database slower. To increase the performance, NoSQL databases does not support ACID properties. They are often faster due to their simpler data models.  
	
	\par		
	There are some research and analysis between NoSQL and RDBMS. \citet{nance2013nosql} had examined the advantages and disadvantages of NoSQL and RDBMS, \cite{cattell2011scalable} analyze similarities between scalable SQL and NoSQL databases, \cite{hadjigeorgiou2013rdbms}  compare the performance between MySQL cluster as RDBMS and Mongodb as NoSQL database.  There are also some researches related to RDBMS with XML or XML database(~\citet{jiang2002xparent}, \citet{shanmugasundaram1999relational}). But there is not much research in the field of NoSQL and XML database together. So this thesis focuses on: Migration of data from XML database to NoSQL database and  performance of both systems.	
	
	\section{Contribution}
		%\input{includes/1/contribution}
		The main contribution of this thesis is that it provides the necessary techniques and algorithms to migrate data from an XML database to NoSQL databases. 
	    
	    ......
	    
	    %More specifically, it will focus on databases like MongoDB, Couchbase and RethinkDB as NoSQL database and BaseX as XML Database. To approach this challenge, it is first necessary to understand the general architecture and data model of each of these databases as well as the way how they are queried.The performance comparison of these two systems will be based on XMARK dataset and its 20 standard queries~\citep{xmark/original}. These 20 queries of XMARK project will be translated to each of NoSQL databases.
		
		
		%This thesis analyze the efficiency of XML and NoSQL.  Moreover, it explains the different database like MongoDB, Couchbase and RethinkDB as NoSQL databases and BaseX as XML Database. To approach this challenge, (? Not clear about the challenge) it is necessary to understand the general architecture and data model of each databases and queried method. The performance comparison of XML and NoSQL method is also carried out based on XMARK dataset and its 20 standard queries~\citep{xmark/original}. These 20 queries of XMARK project is translated to each of NoSQL databases.
	\section{Overview }
		This master thesis is organize as follow: In chapter~\ref{nosql-xml-database}, we will introduce the two new databases systems. Chapter~\ref{semi-structure-data} defines the techniques and necessary algorithms to convert XML  to JSON. In same section, the XMARK dataset will be introduced and procedure to store in NoSQL databases. In Chapter~\ref{ch:benchmarking}, the spotlight is on the performance tests and comparative analysis of each of the NoSQL databases with BaseX, an native XML database, based on the XMark benchmarking project.		
	