%\section{Introduction}
%Data is growing very rapidly recent years. The huge amount of data collected from different field like business, medical and scientific research centers and the web etc. produces data that do not follow the specified format. Such data  is the main causes of this growth. Traditional data management systems are designed to store and process the data that To handle such unstructured data two new database system XML DBMS and NoSQL database system came to exists and success in their way. In this thesis we will study the similarities of these two new DBMS and the way they manage data. 

\section{Motivation}
		\label{motivation}
		    As the digital world growing very fast, the massive amount of unstructured semi-structured data collected from various fields is becoming more complex for storage and querying. The traditional data management tools such as relational database management systems(RDBMS) are mainly designed for structured data. That is, structured data has fixed data model like rows and columns which perfectly fits into relational database. On the other hand, semi-structured data contains different data types and there is no separation between data and schema which is completely irrelevant to relational databases. 
\par		
Relational database works best with the well organized tables. The structure of the data is pre-defined by the layout of the table and the types of columns. This pre-design rigid schema structure of RDBMS has added the complexities to process such variety of data. It is possible to scale relational databases, but to scale beyond a point, it must be distributed across multiple servers. Joining the tables across distributed systems is difficult task. It is slow and inefficient, when dealing with complex data structures that does not fit into a table properly. RDBMS also offers a big set of features that are not oftenly need for all users.
	\par
To overcome the limitations of RDBMS, NoSQL and XML databases are used to store and process unstructured and semi-structured data. These databases supports horizontal scalability, hence the data is distributed in different nodes. As relational databases support ACID properties, there are several restrains to perform on every piece of the data which makes the database slower. To increase the performance and scalability, NoSQL databases often provides weak consistency like eventual consistency instead of ACID guarantees. Beside that,  These database are often faster due to their simple simple data model and selective features.
	\par		
	There are some research and analysis between NoSQL and RDBMS. \citet{nance2013nosql} had examined the advantages and disadvantages of both NoSQL and RDBMS, the similarities between scalable SQL and NoSQL databases has been analyzed in \cite{cattell2011scalable}. The performance of MongoDB as NoSQL and MySQL cluster as  RDBMS has been compared in   \cite{hadjigeorgiou2013rdbms}.  There are also some researches related to RDBMS with XML or XML database like ~\cite{jiang2002xparent} and \cite{shanmugasundaram1999relational}).
	%There is not much research in the field of NoSQL and XML databases together.
	In this thesis, we are going to analyzed the NoSQL databases and XML databases together. First, the data from an XML database will be migrated to the NoSQL databases and the performance of both systems will be examined
\par 
More specifically, it will focus on databases like MongoDB, Couchbase and RethinkDB as NoSQL databases and BaseX as XML database. To approach this challenge, it is first necessary to understand the general architecture and data model of each of these databases as well as the way how they are queried. The performance comparison of these two systems will be based on XMARK dataset and its 20 standard queries~\citep{xmark/original}. These 20 queries of XMARK project will be translated to each of NoSQL databases.
	
	\section{Contribution}
		%\input{includes/1/contribution}
		The main contribution of this thesis is to migrate data based from an XML database to three document-oriented NoSQL databases: MongoDB, Couchbase and RethinkDB. The XQuery expressions of standard XML benchmarking project XMark will be  translated  into each of NoSQL databases and their performances are  benchmarked according to the data their model. XML and JSON are the data format stored and queried by an XML database and document-oriented NoSQL databases respectively. Therefore, this thesis also provides necessary techniques and algorithms to convert data from XML to JSON format.
	    
	    %More specifically, it will focus on databases like MongoDB, Couchbase and RethinkDB as NoSQL database and BaseX as XML Database. To approach this challenge, it is first necessary to understand the general architecture and data model of each of these databases as well as the way how they are queried.The performance comparison of these two systems will be based on XMARK dataset and its 20 standard queries~\citep{xmark/original}. These 20 queries of XMARK project will be translated to each of NoSQL databases.
		
		
		%This thesis analyze the efficiency of XML and NoSQL.  Moreover, it explains the different database like MongoDB, Couchbase and RethinkDB as NoSQL databases and BaseX as XML Database. To approach this challenge, (? Not clear about the challenge) it is necessary to understand the general architecture and data model of each databases and queried method. The performance comparison of XML and NoSQL method is also carried out based on XMARK dataset and its 20 standard queries~\citep{xmark/original}. These 20 queries of XMARK project is translated to each of NoSQL databases.
	\section{Overview }
		This master thesis is organized as follows: In chapter~\ref{nosql-xml-database}, we will introduce the two new databases systems with their data and query model, indexing methods etc. Chapter~\ref{semi-structure-data} defines the techniques and necessary algorithms to convert XML  to JSON. In same chapter, the XMARK dataset will be introduced and we will contribute the procedure to store XML data into the NoSQL databases. In Chapter~\ref{ch:benchmarking}, the spotlight is on the performance tests and analysis of each of the NoSQL databases and BaseX.