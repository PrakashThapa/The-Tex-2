For benchmarking, we have used same hardware configuration for all four databases in standalone mode. All the tests in this chapter is done using  Linux 3.13.0-49 (Ubuntu 14.04, 64bit) with Intel Core i3 with 4GB RAM and 256GB SSD machine.
Table~\ref{benchmark-configuration-table} describes hardware and software information used for benchmarking. 
\begin{table}[h]	
	\centering
	\caption{Reportable Benchmark Information}
	\begin{tabular}{|c|c|c|c} 
		\hline
		Hardware: & \multicolumn{3}{|c|}{CPU: Intel core i3, RAM: 4GB, SSD:256GB  } \\
		\hline
		Software: & \multicolumn{3}{|c|}{OS: Linux 3.13.0-49(Ubuntu 14.04), 64 bit} \\
		\hline
		DBMS and Version: & \multicolumn{3}{|c|}{ BaseX: 8.1, MongoDB:2.6.9, RethinkDB:2.0,Couchbase Server:3.0 } \\
		\hline
		Other Software: & \multicolumn{3}{|c|}{Nodejs: v0.12.2, Java:1.7.0\_80 } \\
		\hline
	\end{tabular}	
	\label{benchmark-configuration-table}
\end{table}
There are certain queries that are dependent on drivers. Those queries are tested through the client-server architecture for all databases. MongoDB, Couchbase and RethinDB natively support Node.js and JavaScript, therefore, these queries are examined through Nodejs version  v0.12.2. For BaseX, the client-server architecture is used with Java 1.7.0. Basex 8.1, MongoDB 2.6.9, Couchbase 3.0 and RethinkDB 2.0 are the versions of databases used for the comparison. 
\par
\section{Database sizes and measurements}\label{benchmark-database-size}
In order to measure the performance of database management systems, we have selected six different size of XMark dataset based on the \textit{xmlgen} factors 0.001, 0.01, 0.1, 1, 10 and 20 that generates the dataset of 111 KB, 1.1 MB, 11 MB, 111MB, 1.1GB and 2.2 GB respectively. These XML datasets are migrated to NoSQL database according to the Section~\ref{xmark-nosql}. For BaseX, database is directly created using import method. The size of the database varies with each of these individual systems. The database systems are used in standalone mode for benchmarking.