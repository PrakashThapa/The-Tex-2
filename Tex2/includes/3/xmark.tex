The XML benchmarking project XMARK~\citep{xmark/original} is one of the most popular and commonly used XML benchmarking projects to date. It provides a small executable tool called \textit{xmlgen} that can be used to create a synthetic XML dataset based on a fixed schema describing an Internet auctions database. xmlgen can be used to build a single record with a large, hierarchical XML tree structure. A factor is specified to scale the generated data, ranging from a few kilobytes to an arbitrary size, limited by the capacity of the system. The textual part of the resulting XML document is constructed from 17,000 most frequently occurring words of Shakespeare's plays.

\subsection{Dataset}\label{xmark-dataset}
The main entities of XMark data are in two groups. The first group consists of  \textit{person}, \textit{open\_auction}, \textit{closed\_auction}, \textit{item} and \textit{category}. The second group's entities \textit{annotation} and \textit{description} are natural language text and document-centric element structure. The relationship between the entities in the first group are expressed as reference and the second group entities are embedded into subtree of first group entities. Figure~\ref{fig:xmark-tree} shows the XMark dataset with the following properties:
\label{xmark:desc:each}
\begin{itemize}	
	\item
	\textit{people} is a collection of \textit{person} element that is connected to buyer and seller of \textit{open\_auctions}, \textit{closed\_auctions}, etc. Each person has an unique identifier \textit{id} to reference to another entities like \textit{open\_auctions} and \textit{closed\_auctions}.
	\item
	\textit{regions} is a collection world regions \textit{africa}, \textit{asia}, \textit{australia}, \textit{europe}, \textit{namerica} and \textit{samerica}. Each of these region has \textit{item} elements which are the objects for sale or already sold. Each \textit{item} carries a unique identifier \textit{id} and has properties like name, payment information, description and a reference to the seller that are encoded as elements. 
	\item
		\textit{open\_auctions} refers to current auctions that contains bid history(increase/decrease over time) with references to bidders and sellers, current bid, the time interval of bid accepted, the status of the transaction, a reference to the item being sold etc.
	\item
		\textit{closed\_auctions} contains auctions that are successfully completed. They have properties like buyer and seller information reference to \textit{person}, a reference to sold items, amount of price, quantity of sold items, date of transaction, type of transaction, and much more.
	\item 
		\textit{categories} is used to classify the items. Each category has a unique identifier used to reference an item, a name and a description.
	\item
		  A \textit{catgraph} links categories into a network.  The full semantics of the XMark dataset can be found in~\cite{xmark/original}.
\end{itemize}
The full ER-Diagram of XMark dataset is illustrated in Fig.~\ref{fig:xmark-schema}. 
\todo{explain the figure}
\begin{figure}[h]

	\centering
	\subfloat[Reference in \textit{XMark}]{
		\includegraphics[width=.5\textwidth]{img/xmark/xmark-ref}{ %xmark-references.png
			\label{fig:xmark-reference}
		}
	}
	\\
	\centering
	\subfloat[Reference in \textit{XMark} dataset tree]{
		\includegraphics[width=0.95\textwidth]{img/xmark/xmark-tree-1.png}{
			\label{fig:xmark-tree}
		}
	}
	\caption{XMark data tree and reference~\citep{xmark/original}}
	\label{fig:xmark-tree-reference}
\end{figure}


\begin{figure}[h]
	\centering
	\includegraphics[width=0.90\textwidth]{img/xmark-schema-4}
	\caption{XMark ER-Diagram. Nodes, solid arrows, and dashed arrows represent schema elements (or attributes, with prefix '@'), structural links, and value links, respectively. Elements with suffix '*' are of SetOf type\citep{xmark/schema-sumerize}}
	\label{fig:xmark-schema}
\end{figure}

\subsection{XMark Queries}\label{xmark-queries}
The XMark project contains XQuery expressions that focuses on the various aspect of language such as aggregation, reference, ordering, wildcard expressions, joins, user defined functions, etc.\citep{xmark/mlynkova2008xml}.The textual representation of 20  XQuery expressions is reprinted in  Table~\ref{tab:xmark-queries}. These queries are divided into different categories  based on the  multiple functionalities of XQuery: 
\begin{enumerate}[label=\arabic*.]
\item  First category tests execution of exact match of string in specified path and consists of only query Q1.
\item It helps to analyze order access of an XML document. Query Q2, Q3 and Q4 are grouped here.
\item Query Q5 evaluates the casting of a value.
\item Queries Q6 and Q7  evaluate regular path expressions.

\item Queries Q8 and Q9 involves to investigate the referencing of a document to another document. 

\item Query Q10 reconstructs the complex results from a result of a query

\item Two queries, Q11 and Q12 are join queries based on values.  The difference between these queries and the reference chasing queries Q8 and Q9 is that references are specified in the DTD and may be optimized with logical IDs.

\item Query Q13  benchmarks the portion reconstruction of original XML document.

\item In Q14 full text search is implemented using single word.

\item The purpose of queries Q15 and Q16 is to observe the path traversals without using wildcards.

\item Query Q17 tests the ability to deal with missing values

\item This category deals with user defined functions and contains query Q18

\item The query Q19 is used to evaluate sorting.

\item Q20 is in last category which observes the  simple aggregation.

\end{enumerate}

\begin {table}[htpb] 
\centering
\caption {The XMark queries. Source:\citep{xmark/original}}
\label {tab:xmark-queries}
\begin{tabular}{r|l}
	\hline
	Q1&Return the name of the person with ID 'person0'.\\
	\hline
	Q2&Return the initial increase of all open auctions.\\
	\hline
	Q3&Return the first and current increase of all open auctions whose current\\
	&increase is at least twice as high as the initial increase.\\
	\hline
	Q4&List the reserves of those open auctions where a certain person issued\\
	&a bid before another person.\\
	\hline
	Q5&How many sold items cost more than 40.\\
	\hline
	Q6&How many items are listed on all continents?\\
	\hline
	Q7&How many pieces of prose are in our database?\\
	\hline
	Q8&List the names of persons and the number of items they bought.\\
	&(Joins person, closed\_auction)\\
	\hline
	Q9&List the names of persons and the names of items they bought in Europe.\\
	&(Joins person\_auction, item)\\
	\hline
	Q10&List all persons according to their interest; use French markup\\
	&in the result.\\
	\hline
	Q11&For each person, list the number of items currently on sale whose\\
	&price does not exceed 0.02\% of the person's income.\\
	\hline
	Q12&For each richer-than-average person, list the number of items currently\\
	&on sale whose price does not exceed 0.02\% of the person's income.\\
	\hline
	Q13&List the names of items registered in Australia along with\\
	&their description.\\
	\hline
	Q14&Return the names of all items whose description contains the word 'gold'.\\
	\hline
	Q15&Print the keywords in emphasis in annotations of closed auctions.\\
	\hline
	Q16&Return the IDs of those auctions that have one or more keywords\\
	&in emphasis.\\
	\hline
	Q17&Which persons don't have a homepage?\\
	\hline
	Q18&Convert the currency of the reserve of all open auctions to\\
	&another currency.\\
	\hline
	Q19&Give an alphabetically ordered list of all items along with their location.\\
	\hline
	Q20&Group customers by their income and output the cardinality of each\\
	&group.\\
	\hline
\end{tabular}
\end {table}
