%\section{Introduction}
%Data is growing very rapidly recent years. The huge amount of data collected from different field like business, medical and scientific research centers and the web etc. produces data that do not follow the specified format. Such data  is the main causes of this growth. Traditional data management systems are designed to store and process the data that To handle such unstructured data two new database system XML DBMS and NoSQL database system came to exists and success in their way. In this thesis we will study the similarities of these two new DBMS and the way they manage data. 

\section{Motivation}
		\label{motivation}
		    As the digital world progresses very fast, the massive amounts of unstructured and semi-structured data collected from various fields are becoming more complex for storage and querying. The traditional data management tools such as relational database management systems(RDBMS) are mainly designed for structured data. This is due to the fact that structured data has fixed data model like rows and columns which perfectly fits into relational database. On the other hand, semi-structured data contains different data types which does not separate data and schema making it unsuitable for RDBMS.
\par		
Relational database works best with the well organized tables. The structure of the data is pre-defined by the layout of the table and the types of columns. This pre-designed rigid schema structure of RDBMS has added the complexities to process such variety of data. It is possible to scale relational databases, but to do so beyond a point requires the data to be distributed across multiple servers. Joining the tables across distributed systems is a challenging task. The process is slow and inefficient when dealing with complex data structures that do not fit into a table properly. RDBMS also offers a big set of features that are not often necessary for all users. \todo{However, they still utilize the resource}
	\par
NoSQL and XML databases are designed to overcome the limitations of RDBMS by storing and processing unstructured and semi-structured data. These databases support horizontal scalability, hence the data is distributed in different nodes. As relational databases guarantee ACID properties, several restrains to perform on every piece of the data delay the database operations. To increase the performance and scalability, NoSQL databases often provides weak consistency like eventual consistency instead of ACID guarantees. Beside that, these database are often faster due to their simple data model and selective features.
	\par		
	There are existing research and analysis between NoSQL and RDBMS. \citet{nance2013nosql} had examined the advantages and disadvantages of both NoSQL and RDBMS, the similarities between scalable SQL and NoSQL databases has been evaluated in \cite{cattell2011scalable}. The performance of MongoDB as NoSQL and MySQL cluster as  RDBMS has been compared in   \cite{hadjigeorgiou2013rdbms}.   Studies have been conducted in the field of RDBMS with XML or XML database in  ~\cite{jiang2002xparent} and \cite{shanmugasundaram1999relational}. Studies involving both NoSQL and XML database are rarely conducted.
	%There is not much research in the field of NoSQL and XML databases together.
\par
In this thesis, we are going to analyze the migration of data and benchmarking of the performance between NoSQL and XML databases. More specifically, it will focus on databases like MongoDB, Couchbase and RethinkDB as NoSQL databases and BaseX as an XML database. To complete this task, it is first necessary to understand the general architecture and data model of each of these databases as well as how they are queried. The performance comparison of these two systems will be based on XMARK dataset and its 20 standard queries~\citep{xmark/original}. These 20 queries of XMARK project will be translated to each of NoSQL databases.
	
	\section{Contribution}
		%\input{includes/1/contribution}
		The main contribution of this thesis is to migrate data based from an XML database to three document-oriented NoSQL databases: MongoDB, Couchbase and RethinkDB. The XQuery expressions of the standard XML benchmarking project known as XMark will be  translated  into each of the NoSQL databases. Their performances are  evaluated according to the their data model. XML and JSON are the data formats stored and queried by XML and document-oriented NoSQL databases respectively. Therefore, this thesis also provides necessary techniques and algorithms to convert data from XML to JSON format.
	    
	    %More specifically, it will focus on databases like MongoDB, Couchbase and RethinkDB as NoSQL database and BaseX as XML Database. To approach this challenge, it is first necessary to understand the general architecture and data model of each of these databases as well as the way how they are queried.The performance comparison of these two systems will be based on XMARK dataset and its 20 standard queries~\citep{xmark/original}. These 20 queries of XMARK project will be translated to each of NoSQL databases.
		
		
		%This thesis analyze the efficiency of XML and NoSQL.  Moreover, it explains the different database like MongoDB, Couchbase and RethinkDB as NoSQL databases and BaseX as XML Database. To approach this challenge, (? Not clear about the challenge) it is necessary to understand the general architecture and data model of each databases and queried method. The performance comparison of XML and NoSQL method is also carried out based on XMARK dataset and its 20 standard queries~\citep{xmark/original}. These 20 queries of XMARK project is translated to each of NoSQL databases.
	\section{Overview }
		This thesis is organized as follows: In chapter~\ref{nosql-xml-database}, we will introduce the two new databases systems with their data and query model, indexing methods etc. Chapter~\ref{semi-structure-data} defines the techniques and necessary algorithms to convert XML to JSON. In the same chapter, the XMARK dataset will be introduced and the procedure to store XML data into the NoSQL databases will be explained. In Chapter~\ref{ch:benchmarking}, the spotlight is on the performance tests and analysis of each of the NoSQL databases and BaseX.