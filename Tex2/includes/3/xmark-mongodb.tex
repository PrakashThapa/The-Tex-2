In MongoDB, collections consist of group of documents with similar structure. Therefore, the data modeling concept of Section~\ref{xmark-nosql} has to be modified marginally. The documents are grouped with their \textit{doctype} from Section~\ref{xmark-nosql}. Each \textit{doctype} represent a collection, there are all together 6 collections. The field \textit{doctype} is already represented as collections, it is removed from all documents.  
For \textit{item} entity,  field \textit{regions} does not change. The \textit{\_id} is the primary index of a document in MongoDB, the identifier attribute of the XMark data \textit{id} is renamed to \textit{\_id} for default indexing.  \textit{closed\_auctions} and \textit{catgraph} do not have an identifier attribute \textit{id}, therefore, system will automatically generate the \textit{\_id} in these collections.
A typical example of MongoDB document for person with identifier \textit{person0} is given in Figure~\ref{code:mongodb-person0}.	

\newbox\mongodbXmarkDocument
\begin{lrbox}{\mongodbXmarkDocument}
\begin{lstlisting}[language=JSON, basicstyle =\scriptsize]
    {
    	<@\textbf{"\_id": "person0",}@>
    	"name": "Kasidit Treweek",
    	"emailaddress": "mailto:Treweek@cohera.com",
    	"phone": "+0 (645) 43954155",
    	"homepage": "http://www.cohera.com/~Treweek",
    	"creditcard": "9941 9701 2489 4716",
    	"profile": {
    		"income": 20186.59,
    		"interest": [
    			{"category": "category251"},
    			{"category": "category341"}
    			],
    		"education": "Graduate School",
    		"business": "No"
    	}
    }
\end{lstlisting}
\end{lrbox}


\newbox\mongodbXmarkChart
\begin{lrbox}{\mongodbXmarkChart}
\begin{tikzpicture}[grow'=right,level distance=1.25in,sibling distance=.25in, font=\scriptsize]
\tikzset{edge from parent/.style= 
            {thick, draw, edge from parent fork right},
         every tree node/.style=
            {draw,minimum width=1in,text width=1in,align=center}}
\Tree 
    [. Database 
        [.{regions}
            [.{... } ]
        ]
        [.people
            [.{person0 } ]
        ] 
        [.{open\_auctions}
            [.{... } ]
        ]
        [.{closed\_auctions}
            [.{... } ]
        ]
        [.{catgraph}
            [.{... } ]
        ]
        [.{catgraph}
            [.{... } ]
        ]
    ]
    
\end{tikzpicture}
\end{lrbox}

\begin{figure}[hbtp]
\centering
\subfloat[Database, collections and documents in MongoDB] {
    \usebox\mongodbXmarkChart
    \label{xmark-mongodb-tree}
}
\\
\centering
\subfloat[{MongoDB document of XMark data in \textit{people} collection} ] {
        \usebox\mongodbXmarkDocument
        \label{code:mongodb-person0}
}

\caption{XMark data in MongoDB}
\label{xmark-mongodb-figure}
\end{figure}