RethinkDB splits data into tables similar to MongoDB's collection.  Each \textit{doctype} of \ref{xmark-nosql} is represented as an individual table. The tables \textit{regions}, \textit{people}, \textit{open\_auctions}, \textit{closed\_auctions}, \textit{catgraph} and \textit{categories} contains the respective documents as of \textit{doctype} in \ref{xmark-nosql}. Hence the attribute \textit{doctype} is removed from all documents.  \textit{id} is the primary key and any document without \textit{id} field is automatically added an identifier during the time of insertion. In Code~\ref{code:rethindb-person0}, an illustration of XMark's person with id \textit{person0} is given
\begin{figure}
\centering
\begin{lstlisting}[language=JSON,basicstyle =\scriptsize]

	{
		"id": "person0",
		"name": "Kasidit Treweek",
		"emailaddress": "mailto:Treweek@cohera.com",
		"phone": "+0 (645) 43954155",
		"homepage": "http://www.cohera.com/~Treweek",
		"creditcard": "9941 9701 2489 4716",
		"profile": {
			"income": 20186.59,
			"interest": [{
				"category": "category251"
			}],
			"education": "Graduate School",
			"business": "No"
		}
	}
\end{lstlisting}
\caption{RethinkDB document \textit{person0} in people table}
\label{code:rethindb-person0}
\end{figure}
	
\begin{comment}

\begin{tikzpicture}[
scale = 1.5, transform shape, thick,
every node/.style = {draw, circle, minimum size = 10mm},
grow = down,  % alignment of characters
level 1/.style = {sibling distance=3cm},
level 2/.style = {sibling distance=4cm}, 
level 3/.style = {sibling distance=2cm}, 
level distance = 1.25cm
]
\node[fill = gray!40, shape = rectangle, rounded corners,
minimum width = 6cm, font = \sffamily] {Coin flipping} 
child { node[shape = circle split, draw, line width = 1pt,
	minimum size = 10mm, inner sep = 0mm, font = \sffamily\large,
	rotate=30] (Start)
	{ \rotatebox{-30}{H} \nodepart{lower} \rotatebox{-30}{T}}
	child {   node [head] (A) {}
		child { node [head] (B) {}}
		child { node [tail] (C) {}}
	}
	child {   node [tail] (D) {}
		child { node [head] (E) {}}
		child { node [tail] (F) {}}
	}
};

% Filling the root (Start)
\begin{scope}[on background layer, rotate=30]
\fill[head] (Start.base) ([xshift = 0mm]Start.east) arc (0:180:5mm)
-- cycle;
\fill[tail] (Start.base) ([xshift = 0pt]Start.west) arc (180:360:5mm)
-- cycle;
\end{scope}

% Labels
\begin{scope}[nodes = {draw = none}]
\path (Start) -- (A) node [near start, left]  {$0.5$};
\path (A)     -- (B) node [near start, left]  {$0.5$};
\path (A)     -- (C) node [near start, right] {$0.5$};
\path (Start) -- (D) node [near start, right] {$0.5$};
\path (D)     -- (E) node [near start, left]  {$0.5$};
\path (D)     -- (F) node [near start, right] {$0.5$};
\begin{scope}[nodes = {below = 11pt}]
\node [name = X] at (B) {$0.25$};
\node            at (C) {$0.25$};
\node [name = Y] at (E) {$0.25$};
\node            at (F) {$0.25$};
\end{scope}
\draw[densely dashed, rounded corners, thin]
(X.south west) rectangle (Y.north east);
\end{scope}
\end{tikzpicture}
\end{comment}