%motivation
As the digital world growing very fast, the massive amount of data collected today in the field varying from business to scientific research, is becoming complex for storage, query and providing mechanism for failover.  As the data collection grows so largely, the traditional data management tools e.g. relational database management systems(RDBMS) struggling effectively handle such data. The pre-design rigid schema structure of RDBMS made more complex for variety of data.  High data velocity where massive read and write operation possible different geographic location, storage of structured, semi-unstructured and and unstructured data and massive volume, together termed as  \textit{Big Data} became a global phenomena which added more complexities on RDMS.
\paragraph{}
New database technologies such as  NoSQL and  XML Databases came in existence to overcome above mentioned problem of RDBMS.  There are many Research and comparative analysis between RDBMS and NoSQL. \citep{xmark/original}

A NoSQL database is non-relational and largely distributed system. XML databases store document-centric XML data. 
